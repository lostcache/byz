\documentclass[11pt]{article}
\usepackage[a4paper,margin=1.5in]{geometry}
\usepackage{inconsolata}  % Monospace font
\usepackage{setspace}
\onehalfspacing  % Increased line spacing
\usepackage{hyperref}
\usepackage{graphicx}
\usepackage{fancyhdr}
\usepackage{minted} % For code snippets; requires -shell-escape
\usepackage{tikz}
\usetikzlibrary{arrows.meta, positioning, calc, shapes.geometric}
\usepackage[table]{xcolor}
\usepackage{array, rotating, colortbl, booktabs}
\usepackage{adjustbox}
\usepackage{caption}
\newcommand{\cmdA}{\ensuremath{\mathsf{A}}} % Command: Attack
\newcommand{\cmdR}{\ensuremath{\mathsf{R}}} % Command: Retreat
\newcommand{\loyal}{\ensuremath{\mathcal{L}}}
\newcommand{\traitor}{\ensuremath{\mathcal{T}}}
\newcommand{\gen}[1]{\ensuremath{G_{#1}}}
% --- Definitions for storing and measuring the table and labels ---
\newsavebox{\mainTableBox}
\newlength{\mainTableHeight}
\newlength{\mainTableWidth}
\newlength{\senderLabelAreaHeight}
\newlength{\majorityLabelAreaHeight}
\usepackage{enumitem}
\usepackage{datetime2}
\usepackage{ragged2e}

% Header and Footer - Blog style
\pagestyle{fancy}
\fancyhf{}
\fancyhead[L]{\texttt{\large Byzantine Bytes}}
\fancyhead[C]{}
\fancyhead[R]{\texttt{\thepage}}
\renewcommand{\headrulewidth}{0.5pt}
\setlength{\headheight}{15pt}

% Metadata
\title{\texttt{\LARGE Byzantine Generals' Problem: \\[0.3em] \Large Decrypting Leslie Lamports' Solution}}
% \author{\texttt{\large A Technical Blog Post}}
% \date{\texttt{\large \today}}
\date{}

\begin{document}

\maketitle

\begin{center}
\begin{minipage}{0.85\textwidth}
\texttt{\large This blog explains Leslie Lamport's solution to the Byzantine Generals' problem that uses unsigned messages, providing a step-by-step walkthrough with visual examples.}
\end{minipage}
\end{center}
\vspace{1em}

\section*{\texttt{\Large Motivation}}
\justifying
Distributed systems are everywhere in our modern digital infrastructure—from cloud computing to blockchain networks. However, a fundamental challenge in these systems is achieving consensus when some components might be faulty or malicious.

The Byzantine Generals' Problem, introduced by Leslie Lamport, Robert Shostak, and Marshall Pease in 1982, elegantly captures this challenge through a military metaphor. Understanding this problem and its solution is crucial for anyone working with distributed systems, blockchain technology, or fault-tolerant computing.

This blog focuses specifically on the solution using unsigned messages—the most challenging scenario where messages carry no authentication. We'll examine a non-trivial instance with the most adversarial case: where the commander is a traitor and strategically sends conflicting commands to create maximum confusion.

The unsigned message solution is particularly fascinating because it operates under the loosest guarantees. Unlike solutions that rely on cryptographic signatures or other forms of authentication, this approach must achieve consensus through pure logic and majority reasoning, making it both intellectually stimulating and practically relevant.
\section*{\texttt{\Large Introduction}}
\justifying
Imagine a scenario where several Byzantine generals have surrounded an enemy city. They must decide collectively whether to $Attack$ or $Retreat$. To succeed, all loyal generals must agree on the same plan of action—either all $Attack$ or all $Retreat$. A split decision would be disastrous.

The generals can only communicate through messengers, and some generals might be traitors (potentially including the commanding general) who seek to prevent the loyal generals from reaching agreement. These traitors can:
\begin{itemize}
    \item Send different messages to different generals
    \item Collude with other traitors
    \item Lie about messages they claim to have received
\end{itemize}

The challenge is to design an algorithm that ensures all loyal generals reach the same decision despite these traitors. This is the essence of the Byzantine Generals' Problem.

The solution we explore does not rely on message signatures or authentication. Instead, it uses a recursive message-passing approach called Oral Message (OM) algorithm that can withstand up to $t$ traitors in a system with at least $3t+1$ generals.

\section*{\texttt{\Large Terminology}}
\justifying
Before diving into the solution, let's establish some terminology:

\begin{itemize}
    \item \textbf{Generals}: The participants in the system, denoted as $G_i$ where $i$ is the general's identifier.
    \item \textbf{Commander}: The general who initiates the decision process (usually $G_0$).
    \item \textbf{Lieutenants}: All other generals who must follow the commander's decision.
    \item \textbf{Loyal ($\mathcal{L}$)}: Generals who follow the algorithm correctly.
    \item \textbf{Traitors ($\mathcal{T}$)}: Generals who may behave arbitrarily and maliciously.
    \item \textbf{Commands}: The possible decisions, typically $Attack$ or $Retreat$.
    \item \textbf{OM($m$)}: The Oral Message algorithm with recursion depth $m$.
\end{itemize}

The Byzantine Generals' Problem has two key requirements:
\begin{enumerate}
    \item \textbf{Agreement}: All loyal generals must agree on the same decision. On receiving equal number of $Attack$ and $Retreat$ messages the loyal generals default to predetermined decision, $Retreat$
    \item \textbf{Validity}: If the commander is loyal, all loyal generals must follow the commander's order.
\end{enumerate}

Lamport proved that with oral (unsigned) messages, we need at least $3t+1$ generals to tolerate $t$ traitors, and the algorithm requires $t+1$ rounds of message exchanges.

\section*{\texttt{\Large The Algorithm}}
\justifying
The Oral Message algorithm (OM) works recursively:

\textbf{OM(0)}: The commander sends a value to all lieutenants, and each lieutenant uses the value received.

\textbf{OM($m$)}, for $m > 0$:
\begin{enumerate}
    \item The commander sends a message to each lieutenant.
    \item Each lieutenant acts as a commander in the OM($m-1$) algorithm to share the message they received with all other lieutenants.
    \item Each lieutenant determines the majority value reported for each commander and uses that as their final decision.
\end{enumerate}

The algorithm guarantees correct operation if $n \geq 3t+1$ where $n$ is the total number of generals and $t$ is the maximum number of traitors.
The recursive nature of this algorithm creates a tree of message exchanges that grows exponentially with $m$. For a system with $n$ generals, the message complexity is $O(n^{m+1})$, which makes it practical only for small values of $t$ and thus $m$.

\section*{\texttt{\Large The Problem}}
\justifying
Let's examine a specific instance of the Byzantine Generals' Problem:

\begin{itemize}
    \item \textbf{Total generals}: $N = 7$
    \item \textbf{Traitors}: $t = 2$
    \item \textbf{Loyal generals}: $N - t = 5$
\end{itemize}

In our scenario:
\begin{itemize}
    \item Commander $G_0$ is a traitor
    \item Lieutenant $G_1$ is also a traitor
    \item Lieutenants $G_2$ through $G_6$ are loyal
\end{itemize}

This is a particularly challenging case because the commander, who initiates the algorithm, is malicious. The commander's strategy is to create maximum confusion by sending conflicting commands:
\begin{itemize}
    \item Commander $G_0$ sends Attack (A) to generals $G_2$, $G_4$, and $G_6$
    \item Commander $G_0$ sends Retreat (R) to generals $G_1$, $G_3$, and $G_5$
\end{itemize}

Additionally, the second traitor $G_1$ will also attempt to confuse the loyal generals by inconsistently relaying messages during the second round.

With $t = 2$ traitors, we need at least $3t+1 = 7$ generals, which is exactly what we have. According to the theory, OM(2) should be sufficient to achieve consensus among loyal generals. Let's see how this works in practice.

\section*{\texttt{\Large The Algorithm in Play}}

\subsection*{\texttt{\large Round 0}}
\justifying
In round 0 of the algorithm, the commander $\gen{0}$ sends commands to all lieutenants. Since the commander is a traitor, it can send different commands to different lieutenants. Figure~\ref{fig:traitor_commander_round_2} illustrates what happens in this round.


% TikZ styles for Byzantine Generals figures
\tikzstyle{general} = [circle, draw, minimum size=0.9cm, font=\small]
\tikzstyle{messageA} = [-{Stealth[length=2.5mm, width=2mm]}, blue, thick] % Blue for Attack
\tikzstyle{messageR} = [-{Stealth[length=2.5mm, width=2mm]}, red, thick]  % Red for Retreat
\tikzstyle{nodeLabel} = [font=\scriptsize, right=1pt]

% Round 1 Figure
\begin{center}
\begin{tikzpicture}[scale=1, transform shape]
    % Commander
    \node[general, fill=red!20, label={[nodeLabel]right:{Commander}}] (g0) at (0,4) {$\gen{0}(\traitor)$};

    % Lieutenants
    \node[general, fill=red!20] (g1) at (-5,0) {$\gen{1}(\traitor)$};
    \node[general, fill=green!20] (g2) at (-3,0) {$\gen{2}(\loyal)$};
    \node[general, fill=green!20] (g3) at (-1,0) {$\gen{3}(\loyal)$};
    \node[general, fill=green!20] (g4) at (1,0) {$\gen{4}(\loyal)$};
    \node[general, fill=green!20] (g5) at (3,0) {$\gen{5}(\loyal)$};
    \node[general, fill=green!20] (g6) at (5,0) {$\gen{6}(\loyal)$};

    % Messages from G0
    \draw[messageR] (g0) -- (g1) node[midway, above, sloped, black]{\cmdR}; % G0 to G1 (Traitor)
    \draw[messageA] (g0) -- (g2) node[midway, above, sloped, black]{\cmdA}; % G0 to G2 (Loyal)
    \draw[messageR] (g0) -- (g3) node[midway, above, sloped, black]{\cmdR}; % G0 to G3 (Loyal)
    \draw[messageA] (g0) -- (g4) node[midway, above, sloped, black]{\cmdA}; % G0 to G4 (Loyal)
    \draw[messageR] (g0) -- (g5) node[midway, above, sloped, black]{\cmdR}; % G0 to G5 (Loyal)
    \draw[messageA] (g0) -- (g6) node[midway, above, sloped, black]{\cmdA}; % G0 to G6 (Loyal)
\end{tikzpicture}
\captionof{figure}{In Round 0 the traitor commander $\gen{0}$ sends inconsistent messages to lieutenants.}
\label{fig:traitor_commander_round_2}
\end{center}

\justifying
As shown in the figure \ref{fig:traitor_commander_round_2}, the traitor commander $\gen{0}$ sends:
\begin{itemize}
    \item $Retreat$ (R) to generals $\gen{1}$ (traitor), $\gen{3}$, and $\gen{5}$
    \item $Attack$ (A) to generals $\gen{2}$, $\gen{4}$, and $\gen{6}$
\end{itemize}

At this point, each lieutenant only knows what the commander told them directly. Table~\ref{tab:traitor_commander_round_1_values} represents this initial knowledge state, with each lieutenant recording the message they received on the diagonal.


\begin{center}
\vspace{1em} % Add some vertical space
\renewcommand{\arraystretch}{1.5}
\setlength{\tabcolsep}{6pt} % Adjusted
\begin{tabular}{c|c|c|c|c|c}
    \hline
    \textbf{$\gen{1}(\traitor)$} & \textbf{$\gen{2}(\loyal)$} & \textbf{$\gen{3}(\loyal)$} & \textbf{$\gen{4}(\loyal)$} & \textbf{$\gen{5}(\loyal)$} & \textbf{$\gen{6}(\loyal)$} \\
    \hline
    \cellcolor{yellow!30}\textbf{\textcolor{red}{\cmdR}} & & & & & \\
    \hline
     & \cellcolor{yellow!30}\textbf{\textcolor{blue}{\cmdA}} & & & & \\
    \hline
     & & \cellcolor{yellow!30}\textbf{\textcolor{red}{\cmdR}} & & & \\
    \hline
     & & & \cellcolor{yellow!30}\textbf{\textcolor{blue}{\cmdA}} & & \\
    \hline
     & & & & \cellcolor{yellow!30}\textbf{\textcolor{red}{\cmdR}} & \\
    \hline
     & & & & & \cellcolor{yellow!30}\textbf{\textcolor{blue}{\cmdA}} \\
    \hline
\end{tabular}
\captionof{table}{Diagonal cells show the message that each lieutenant ($\gen{1}$ - $\gen{6}$) receives from the traitor commander ($\gen{0}$). Off-diagonal cells are empty, to be filled by messages exchanged between lieutenants in the next rounds.}
\label{tab:traitor_commander_round_1_values}
\end{center}

\subsection*{\texttt{\large Round 1}}
\justifying
In this round each lieutenant takes turns acting as the commander for the round and tells all other lieutenants the message they received from the commander of the previous round (In this case Round 0 and the original commander $\gen{0}$). Table \ref{tab:round_1_iter} covers all the iteration of Round 1.

\noindent
\begin{center}
\begin{tabular}{ccc}
\toprule
\textbf{Iter No} & \textbf{Acting Commander} & \textbf{Lieutenants} \\
\midrule
1 & $\gen{1}$ & $\gen{2}, \gen{3}, \gen{4}, \gen{5}, \gen{6}$ \\
2 & $\gen{2}$ & $\gen{1}, \gen{3}, \gen{4}, \gen{5}, \gen{6}$ \\
3 & $\gen{3}$ & $\gen{1}, \gen{2}, \gen{4}, \gen{5}, \gen{6}$ \\
4 & $\gen{4}$ & $\gen{1}, \gen{2}, \gen{3}, \gen{5}, \gen{6}$ \\
5 & $\gen{5}$ & $\gen{1}, \gen{2}, \gen{3}, \gen{4}, \gen{6}$ \\
6 & $\gen{6}$ & $\gen{1}, \gen{2}, \gen{3}, \gen{4}, \gen{5}$ \\
\bottomrule
\end{tabular}
\captionof{table}{Displays all the scenarios for the Round 1 of the example}
\label{tab:round_1_iter}
\end{center}


Note that each lieutenant while acting as the commander does not send out any new commands, but instead just reports what they claim the commander of the previous round (In this case, the original commander $\gen{0}$) sent them.

Figure~\ref{fig:traitor_commander_traitor_round_2} illustrates a scenario where a traitor lieutenant, In our example $\gen{1}$, while acting as the commander for the round, sends conflicting messages.


% TikZ styles for Byzantine Generals figures
\tikzstyle{general} = [circle, draw, minimum size=0.9cm, font=\small]
\tikzstyle{messageA} = [-{Stealth[length=2.5mm, width=2mm]}, blue, thick] % Blue for Attack
\tikzstyle{messageR} = [-{Stealth[length=2.5mm, width=2mm]}, red, thick]  % Red for Retreat
\tikzstyle{nodeLabel} = [font=\scriptsize, right=1pt]

% --- Definitions for storing and measuring the table and labels ---
\begin{center}
\begin{tikzpicture}[scale=1, transform shape]
    % Commander
    \node[general, fill=red!20, label={[nodeLabel]right:{Commander}}] (g0) at (0,4) {$\gen{0}(\traitor)$};

    % Lieutenants (Level 1)
    \node[general, fill=red!20] (g1) at (-5,0) {$\gen{1}(\traitor)$};
    \node[general, fill=green!5] (g2) at (-3,0) {$\gen{2}(\loyal)$};
    \node[general, fill=green!5] (g3) at (-1,0) {$\gen{3}(\loyal)$};
    \node[general, fill=green!5] (g4) at (1,0) {$\gen{4}(\loyal)$};
    \node[general, fill=green!5] (g5) at (3,0) {$\gen{5}(\loyal)$};
    \node[general, fill=green!5] (g6) at (5,0) {$\gen{6}(\loyal)$};

    % Messages from Commander (Level 1 edges)
    \draw[messageR] (g0) -- (g1) node[midway, above, sloped, black]{\cmdR};
    \draw[messageA] (g0) -- (g2) node[midway, above, sloped, black]{\cmdA};
    \draw[messageR] (g0) -- (g3) node[midway, above, sloped, black]{\cmdR};
    \draw[messageA] (g0) -- (g4) node[midway, above, sloped, black]{\cmdA};
    \draw[messageR] (g0) -- (g5) node[midway, above, sloped, black]{\cmdR};
    \draw[messageA] (g0) -- (g6) node[midway, above, sloped, black]{\cmdA};

    % Second Level: Lieutenants communicating with each other (add this section)
    \begin{scope}[yshift=-3cm] % Shift downward for the second layer
        % New nodes (spaced horizontally like the first layer)
        \node[general, fill=green!20] (g2a) at (-5,-3) {$\gen{2}(\loyal)$}; % Match g1's x-position
        \node[general, fill=green!20] (g3a) at (-3,-3) {$\gen{3}(\loyal)$}; % Match g2's x-position
        \node[general, fill=green!20] (g4a) at (-1,-3) {$\gen{4}(\loyal)$};
        \node[general, fill=green!20] (g5a) at (1,-3) {$\gen{5}(\loyal)$};
        \node[general, fill=green!20] (g6a) at (3,-3) {$\gen{6}(\loyal)$};

        % Draw edges between lieutenants (example: traitor g1 sends conflicting messages)
        \draw[messageR] (g1) -- (g2a) node[midway, above, sloped, black]{\cmdR};

        \draw[messageA] (g1) -- (g3a) node[midway, above, sloped, black]{\cmdA};
        \draw[messageR] (g1) -- (g4a) node[midway, above, sloped, black]{\cmdR};
        \draw[messageA] (g1) -- (g5a) node[midway, above, sloped, black]{\cmdA};
        \draw[messageR] (g1) -- (g6a) node[midway, above, sloped, black]{\cmdR};
    \end{scope}
\end{tikzpicture}
\captionof{figure}{The figure shows the scenario when  $\gen{1}$ acts as the commander. Being a traitor it deliberately sends contradicting information to other lieutenants.}
\label{fig:traitor_commander_traitor_round_2}
\end{center}

\subsection*{\texttt{\large Round 2}}
Round 2 will take place for every iteration of Round 1 as the lieutenants of Round 1 take turn acting as the commander.

During the final round, Round 2, lieutenants of the round instead of acting as the commander simply gossip about what they received from the commander of the previous round ($\gen{1}$ in our example).

Continuing the example instance we considered for Round 1 where $\gen{1}$ acts as the commander, we will now demonstrate what happened during the instance of Round 2 where $\gen{1}$ is the commander and $\gen{2}, \gen{3}, \gen{4}, \gen{5} \  \& \ \gen{6}$ are the lieutenants.

As $m = 2$ for our example, Round 2 will act as the final round, so instead of taking turns acting as the commander in the recursive fashion like earlier rounds, lieutenants $\gen{2}, \gen{3}, \gen{4}, \gen{5} \  \& \ \gen{6}$ simply relay the message that they received from the commander $\gen{1}$. We visualize the process using Table \ref{tab:traitor_commander_round_0}

\vspace{1em}

% --- Step 1: Typeset the main data table into a savebox to measure its dimensions ---
\sbox{\mainTableBox}{%
  \renewcommand{\arraystretch}{1.3}%
  \setlength{\tabcolsep}{12pt}%
  \begin{tabular}{c|c|c|c|c|c|c}
    \textbf{} & \textbf{$\gen{1}$} & \textbf{$\gen{2}$} & \textbf{$\gen{3}$} & \textbf{$\gen{4}$} & \textbf{$\gen{5}$} & \textbf{$\gen{6}$} \\
    \hline

    \textbf{$\gen{1}$} & \cellcolor{gray!30} & \cellcolor{gray!30} & \cellcolor{gray!30} & \cellcolor{gray!30} & \cellcolor{gray!30} & \cellcolor{gray!30} \\
    \textbf{$\gen{2}$} & \cellcolor{gray!30} & \cellcolor{yellow!30}\textbf{\textcolor{red}{R}} & \cellcolor{red!20}\textcolor{red}{R} & \cellcolor{red!20}\textcolor{red}{R} & \cellcolor{red!20}\textcolor{red}{R} & \cellcolor{red!20}\textcolor{red}{R} \\
    \textbf{$\gen{3}$} & \cellcolor{gray!30} & \cellcolor{blue!20}\textcolor{blue}{A} & \cellcolor{yellow!30}\textbf{\textcolor{blue}{A}} & \cellcolor{blue!20}\textcolor{blue}{A} & \cellcolor{blue!20}\textcolor{blue}{A} & \cellcolor{blue!20}\textcolor{blue}{A} \\
    \textbf{$\gen{4}$} & \cellcolor{gray!30} & \cellcolor{red!20}\textcolor{red}{R} & \cellcolor{red!20}\textcolor{red}{R} & \cellcolor{yellow!30}\textbf{\textcolor{red}{R}} & \cellcolor{red!20}\textcolor{red}{R} & \cellcolor{red!20}\textcolor{red}{R} \\
    \textbf{$\gen{5}$} & \cellcolor{gray!30} & \cellcolor{blue!20}\textcolor{blue}{A} & \cellcolor{blue!20}\textcolor{blue}{A} & \cellcolor{blue!20}\textcolor{blue}{A} & \cellcolor{yellow!30}\textbf{\textcolor{blue}{A}} & \cellcolor{blue!20}\textcolor{blue}{A} \\
    \textbf{$\gen{6}$} & \cellcolor{gray!30} & \cellcolor{red!20}\textcolor{red}{R} & \cellcolor{red!20}\textcolor{red}{R} & \cellcolor{red!20}\textcolor{red}{R} & \cellcolor{red!20}\textcolor{red}{R} & \cellcolor{yellow!30}\textbf{\textcolor{red}{R}} \\
    \hline\hline
    \textbf{Majority} & \cellcolor{gray!30} & \cellcolor{red!75}\textbf{\textcolor{white}{R}}  & \cellcolor{red!75}\textbf{\textcolor{white}{R}}  & \cellcolor{red!75}\textbf{\textcolor{white}{R}}  & \cellcolor{red!75}\textbf{\textcolor{white}{R}} & \cellcolor{red!75}\textbf{\textcolor{white}{R}} \\
  \end{tabular}%
}
% --- Step 2: Get the measured height and width of the saved table ---
\setlength{\mainTableHeight}{\dimexpr\ht\mainTableBox+\dp\mainTableBox\relax}
\setlength{\mainTableWidth}{\wd\mainTableBox}


% --- Step 3: Calculate proportional heights for Sender and Majority label areas ---
% Sender area corresponds to 6 out of 7 conceptual rows.
% Majority area corresponds to 1 out of 7 conceptual rows.
\setlength{\senderLabelAreaHeight}{\dimexpr 6\mainTableHeight / 7 \relax}
\setlength{\majorityLabelAreaHeight}{\dimexpr \mainTableHeight - \senderLabelAreaHeight \relax} % Ensures sum is exactly \mainTableHeight

% --- Step 4: Assemble the labels and table using adjustbox for vertical centering ---

\newlength{\labelcolwidth} % Define a new length
\settowidth{\labelcolwidth}{\textbf{Majority}} % Set it to the width of "Majority"

\centering
\begin{tabular}{@{}c@{\hspace{1em}}c@{}}
  % Top row: Empty space above Sender label + Receiver header
  & {\large \textbf{Receiver}} \\[0.3em]
  % Bottom row: Sender label + Main data table
  \adjustbox{valign=c}{% Vertically center the entire label column box
    \parbox[t][\mainTableHeight][c]{\labelcolwidth}{ % Use \labelcolwidth for the outer parbox width
      \centering % Horizontally center the inner parboxes (Sender & Majority areas)
      \parbox[c][\senderLabelAreaHeight][c]{\labelcolwidth}{ % Sender area, use \labelcolwidth
        \centering\rotatebox[origin=c]{90}{\large\textbf{Sender}}
      }
    }
  }
  &
  \adjustbox{valign=c}{\usebox{\mainTableBox}} % Use the saved (and measured) table, vertically centered
\end{tabular}
\captionof{table}{Round 2 gossip where each cell shows the message the sender tells the receiver that it received from the commander, in this case $\gen{1}$. Yellow diagonal: sender's own message from $\gen{1}$. Grey cells: $\gen{1}$ is the acting commander and does not participate in consensus. Blue = Attack (A), Red = Retreat (R).}
\label{tab:traitor_commander_round_0}

\vspace{1em}
\justifying
In our table visualization, the row headers are marked as $Senders$ and  the column headers are marked as $Receivers$, as each lieutenant gossips by filling out the rows (telling the other lieutenant what it received from the acting commander, $\gen{1}$ in this case). Thus, the diagonals will simply contain the value that the lieutenant received from the commander.

Since each of the lieutenants in Round 2 is loyal, they just send out the exact same message that they received from the acting commander $\gen{1}$

After the gossip is over each lieutenant takes the majority of all the messages that they received during the gossip and treats it as the truth. In Table \ref{tab:traitor_commander_round_0} a column represents the messages that a lieutenant receives from other lieutenants along with the message that they received from the acting commander $\gen{1}$. The values in the last row contains the majority of the messages in that column.

\subsection*{\texttt{\large Back To Round 1}}

The majority that was obtained from the Round 2 during the iteration of Round 1 (In our example, when $\gen{1}$ was acting as the commander) is then used by the lieutenants for the same iteration of Round 1. In the example that we consider the majority that was obtained from Round 2  when the commander was $\gen{1}$ is then used by the lieutenants as the truth values for Round 1 when $\gen{1}$ was acting as the commander. We visualize it using Table \ref{tab:traitor_commander_round_2_summary}

\vspace{0.7em}

\vspace{1em} % Add some vertical space
\renewcommand{\arraystretch}{1.5}
\setlength{\tabcolsep}{6pt} % Adjusted
\begin{tabular}{c|c|c|c|c|c}
    \hline
    \textbf{$\gen{1}(\traitor)$} & \textbf{$\gen{2}(\loyal)$} & \textbf{$\gen{3}(\loyal)$} & \textbf{$\gen{4}(\loyal)$} & \textbf{$\gen{5}(\loyal)$} & \textbf{$\gen{6}(\loyal)$} \\
    \hline
    \cellcolor{yellow!30}\textbf{\textcolor{red}{\cmdR}} & \cellcolor{red!75}\textbf{\textcolor{white}{R}} & \cellcolor{red!75}\textbf{\textcolor{white}{R}} & \cellcolor{red!75}\textbf{\textcolor{white}{R}} & \cellcolor{red!75}\textbf{\textcolor{white}{R}} & \cellcolor{red!75}\textbf{\textcolor{white}{R}} \\
    \hline
    & \cellcolor{yellow!30}\textbf{\textcolor{blue}{\cmdA}} & & & & \\
    \hline
    & & \cellcolor{yellow!30}\textbf{\textcolor{red}{\cmdR}} & & & \\
    \hline
    & & & \cellcolor{yellow!30}\textbf{\textcolor{blue}{\cmdA}} & & \\
    \hline
    & & & & \cellcolor{yellow!30}\textbf{\textcolor{red}{\cmdR}} & \\
    \hline
    & & & & & \cellcolor{yellow!30}\textbf{\textcolor{blue}{\cmdA}} \\
    \hline
\end{tabular}
\captionof{table}{The majority obtained from Round 2, visualized using Table ~\ref{tab:traitor_commander_round_0} is now used as the truth in Round 1, which was visualized using Table ~\ref{fig:traitor_commander_round_2} to get this new Table that represents the knowledge after the completion of an iteration of Round 1 when $\gen{1}$ completes acting as the commander.}
\label{tab:traitor_commander_round_2_summary}

\vspace{1em}
To formalize, \textbf{\textit{The majority from round $O(m)$ is then used as the truth in $O(m - 1)$}}.
\vspace{1em}

Like mentioned earlier, each lieutenant acts as the commander in Round 1 that will result in a similar round 0. where other lieutenants will gossip about the message that this ``new" commander sends them.

\subsection*{\texttt{\large Second Iteration of Round 1}}
For the sake of completeness of the example, we will now carry out the same process for the second iteration of Round 1 when $\gen{2}$ (a Loyal lieutenant) acts as the commander, visualized in Figure \ref{fig:traitor_commander_round_1_iter_2}

% TikZ styles for Byzantine Generals figures
\tikzstyle{general} = [circle, draw, minimum size=0.9cm, font=\small]
\tikzstyle{messageA} = [-{Stealth[length=2.5mm, width=2mm]}, blue, thick] % Blue for Attack
\tikzstyle{messageR} = [-{Stealth[length=2.5mm, width=2mm]}, red, thick]  % Red for Retreat
\tikzstyle{nodeLabel} = [font=\scriptsize, right=1pt]


\begin{center}
\begin{tikzpicture}[scale=1, transform shape]
    % Commander
    \node[general, fill=red!20, label={[nodeLabel]right:{Commander}}] (g0) at (0,4) {$\gen{0}(\traitor)$};

    % Lieutenants (Level 1)
    \node[general, fill=red!5] (g1) at (-5,0) {$\gen{1}(\traitor)$};
    \node[general, fill=green!15] (g2) at (-3,0) {$\gen{2}(\loyal)$};
    \node[general, fill=green!5] (g3) at (-1,0) {$\gen{3}(\loyal)$};
    \node[general, fill=green!5] (g4) at (1,0) {$\gen{4}(\loyal)$};
    \node[general, fill=green!5] (g5) at (3,0) {$\gen{5}(\loyal)$};
    \node[general, fill=green!5] (g6) at (5,0) {$\gen{6}(\loyal)$};

    % Messages from Commander (Level 1 edges)
    \draw[messageR] (g0) -- (g1) node[midway, above, sloped, black]{\cmdR};
    \draw[messageA] (g0) -- (g2) node[midway, above, sloped, black]{\cmdA};
    \draw[messageR] (g0) -- (g3) node[midway, above, sloped, black]{\cmdR};
    \draw[messageA] (g0) -- (g4) node[midway, above, sloped, black]{\cmdA};
    \draw[messageR] (g0) -- (g5) node[midway, above, sloped, black]{\cmdR};
    \draw[messageA] (g0) -- (g6) node[midway, above, sloped, black]{\cmdA};

    % Second Level: Lieutenants communicating with each other (add this section)
    \begin{scope}[yshift=-3cm] % Shift downward for the second layer
        % New nodes (spaced horizontally like the first layer)
        \node[general, fill=red!20] (g1a) at (-5.3,-3) {$\gen{1}(\traitor)$};
        \node[general, fill=green!20] (g3a) at (-1,-3) {$\gen{3}(\loyal)$};
        \node[general, fill=green!20] (g4a) at (1,-3) {$\gen{4}(\loyal)$};
        \node[general, fill=green!20] (g5a) at (3,-3) {$\gen{5}(\loyal)$};
        \node[general, fill=green!20] (g6a) at (5,-3) {$\gen{6}(\loyal)$};

        % Draw edges between lieutenants (example: traitor g1 sends conflicting messages)
        \draw[messageA] (g2) -- (g1a) node[midway, above, sloped, black]{\cmdA};

        \draw[messageA] (g2) -- (g3a) node[midway, above, sloped, black]{\cmdA};
        \draw[messageA] (g2) -- (g4a) node[midway, above, sloped, black]{\cmdA};
        \draw[messageA] (g2) -- (g5a) node[midway, above, sloped, black]{\cmdA};
        \draw[messageA] (g2) -- (g6a) node[midway, above, sloped, black]{\cmdA};
    \end{scope}
\end{tikzpicture}
\label{fig:traitor_commander_round_1_iter_2}
\captionof{figure}{Round 1 where, $\gen{2}$ acts as the commander for and sends the message that it received from the previous commander, $\gen{0}$, The only difference is that $\gen{2}$ being loyal it sends the same message that it received from $\gen{0}$}
\end{center}

\subsection*{\texttt{\large Round 2, Again}}
Like in the previous iteration, the rest of the lieutenants ($\gen{1}, \gen{3}, \gen{4}, \gen{5}\ \&\ \gen{6}$) will gossip about the message that they receive from the new acting commander $\gen{2}$. The only difference is that $\gen{1}$ being a traitor, it will send contradicting messages to other lieutenants, lying about what it received from $\gen{2}$ to create confusion. The instance is visualized using Table \ref{fig:traitor_commander_round_1_second_iter}


\sbox{\mainTableBox}{%
  \renewcommand{\arraystretch}{1.3}%
  \setlength{\tabcolsep}{12pt}%
  \begin{tabular}{c|c|c|c|c|c|c}
    \textbf{} & \textbf{$\gen{1}$} & \textbf{$\gen{2}$} & \textbf{$\gen{3}$} & \textbf{$\gen{4}$} & \textbf{$\gen{5}$} & \textbf{$\gen{6}$} \\
    \hline

    \textbf{$\gen{1}$} & \cellcolor{yellow!30}\textbf{\textcolor{blue}{A}} & \cellcolor{gray!30} & \cellcolor{red!20}\textcolor{red}{R} & \cellcolor{red!20}\textcolor{red}{R} & \cellcolor{red!20}\textcolor{red}{R} & \cellcolor{red!20}\textcolor{red}{R} \\
    \textbf{$\gen{2}$} & \cellcolor{gray!30} & \cellcolor{gray!30} & \cellcolor{gray!30} & \cellcolor{gray!30} & \cellcolor{gray!30} & \cellcolor{gray!30} \\
    \textbf{$\gen{3}$} & \cellcolor{blue!20}\textcolor{blue}{A} & \cellcolor{gray!30} & \cellcolor{yellow!30}\textbf{\textcolor{blue}{A}} & \cellcolor{blue!20}\textcolor{blue}{A} & \cellcolor{blue!20}\textcolor{blue}{A} & \cellcolor{blue!20}\textcolor{blue}{A} \\
    \textbf{$\gen{4}$} & \cellcolor{blue!20}\textcolor{blue}{A} & \cellcolor{gray!30} & \cellcolor{blue!20}\textbf{\textcolor{blue}{A}} & \cellcolor{yellow!30}\textcolor{blue}{A} & \cellcolor{blue!20}\textcolor{blue}{A} & \cellcolor{blue!20}\textcolor{blue}{A} \\
    \textbf{$\gen{5}$} & \cellcolor{blue!20}\textcolor{blue}{A} & \cellcolor{gray!30} & \cellcolor{blue!20}\textcolor{blue}{A} & \cellcolor{blue!20}\textcolor{blue}{A} & \cellcolor{yellow!30}\textbf{\textcolor{blue}{A}} & \cellcolor{blue!20}\textcolor{blue}{A} \\
    \textbf{$\gen{6}$} & \cellcolor{blue!20}\textcolor{blue}{A} & \cellcolor{gray!30} & \cellcolor{blue!20}\textcolor{blue}{A} & \cellcolor{blue!20}\textcolor{blue}{A} & \cellcolor{blue!20}\textbf{\textcolor{blue}{A}} & \cellcolor{yellow!30}\textcolor{blue}{A} \\
    \hline\hline
    \textbf{Majority} & \cellcolor{blue!75}\textbf{\textcolor{white}{A}} & \cellcolor{gray!30} & \cellcolor{blue!75}\textbf{\textcolor{white}{A}}& \cellcolor{blue!75}\textbf{\textcolor{white}{A}} & \cellcolor{blue!75}\textbf{\textcolor{white}{A}} & \cellcolor{blue!75}\textbf{\textcolor{white}{A}} \\
  \end{tabular}%
}
% --- Step 2: Get the measured height and width of the saved table ---
\setlength{\mainTableHeight}{\dimexpr\ht\mainTableBox+\dp\mainTableBox\relax}
\setlength{\mainTableWidth}{\wd\mainTableBox}


% --- Step 3: Calculate proportional heights for Sender and Majority label areas ---
% Sender area corresponds to 6 out of 7 conceptual rows.
% Majority area corresponds to 1 out of 7 conceptual rows.
\setlength{\senderLabelAreaHeight}{\dimexpr 6\mainTableHeight / 7 \relax}
\setlength{\majorityLabelAreaHeight}{\dimexpr \mainTableHeight - \senderLabelAreaHeight \relax} % Ensures sum is exactly \mainTableHeight

% --- Step 4: Assemble the labels and table using adjustbox for vertical centering ---

\newlength{\labelcolwidth} % Define a new length
\settowidth{\labelcolwidth}{\textbf{Majority}} % Set it to the width of "Majority"

\begin{center}
\begin{tabular}{@{}c@{\hspace{1em}}c@{}}
  % Top row: Empty space above Sender label + Receiver header
  & {\large \textbf{Receiver}} \\[0.3em]
  % Bottom row: Sender label + Main data table
  \adjustbox{valign=c}{% Vertically center the entire label column box
    \parbox[t][\mainTableHeight][c]{\labelcolwidth}{ % Use \labelcolwidth for the outer parbox width
      \centering % Horizontally center the inner parboxes (Sender & Majority areas)
      \parbox[c][\senderLabelAreaHeight][c]{\labelcolwidth}{ % Sender area, use \labelcolwidth
        \centering\rotatebox[origin=c]{90}{\large\textbf{Sender}}
      }
    }
  }
  &
  \adjustbox{valign=c}{\usebox{\mainTableBox}} % Use the saved (and measured) table, vertically centered
\end{tabular}
\captionof{table}{Round 0 gossip, again each cell shows the message the sender tells the receiver that it received from the commander, in this case $\gen{2}$. Yellow diagonal: sender's own message from $\gen{2}$. Grey cells: $\gen{2}$ is the acting commander and does not participate in consensus. Blue = Attack (A), Red = Retreat (R).}
\label{fig:traitor_commander_round_1_second_iter}
\end{center}

Once again exactly like before, the lieutenants use the majority of the messages that they receive (represented by column of the table) and this majority is used by the lieutenants for the iteration of Round 1 when $\gen{2}$ was acting as the commander as visualized in Table \ref{tab:traitor_commander_round_1_loyal_commander}


\begin{center}
\vspace{1em} % Add some vertical space
\renewcommand{\arraystretch}{1.5}
\setlength{\tabcolsep}{6pt} % Adjusted
\begin{tabular}{c|c|c|c|c|c}
    \hline
    \textbf{$\gen{1}(\traitor)$} & \textbf{$\gen{2}(\loyal)$} & \textbf{$\gen{3}(\loyal)$} & \textbf{$\gen{4}(\loyal)$} & \textbf{$\gen{5}(\loyal)$} & \textbf{$\gen{6}(\loyal)$} \\
    \hline
    \cellcolor{yellow!30}\textbf{\textcolor{red}{\cmdR}} & \cellcolor{red!75}\textbf{\textcolor{white}{R}} & \cellcolor{red!75}\textbf{\textcolor{white}{R}} & \cellcolor{red!75}\textbf{\textcolor{white}{R}} & \cellcolor{red!75}\textbf{\textcolor{white}{R}} & \cellcolor{red!75}\textbf{\textcolor{white}{R}} \\
    \hline
    \cellcolor{blue!75}\textbf{\textcolor{white}{A}} & \cellcolor{yellow!30}\textbf{\textcolor{blue}{\cmdA}} & \cellcolor{blue!75}\textbf{\textcolor{white}{A}} & \cellcolor{blue!75}\textbf{\textcolor{white}{A}} & \cellcolor{blue!75}\textbf{\textcolor{white}{A}} & \cellcolor{blue!75}\textbf{\textcolor{white}{A}} \\
    \hline
    & & \cellcolor{yellow!30}\textbf{\textcolor{red}{\cmdR}} & & & \\
    \hline
    & & & \cellcolor{yellow!30}\textbf{\textcolor{blue}{\cmdA}} & & \\
    \hline
    & & & & \cellcolor{yellow!30}\textbf{\textcolor{red}{\cmdR}} & \\
    \hline
    & & & & & \cellcolor{yellow!30}\textbf{\textcolor{blue}{\cmdA}} \\
    \hline
\end{tabular}
\label{tab:traitor_commander_round_1_loyal_commander}
\captionof{table}{The majority obtained from Round 0, visualized using Table \ref{fig:traitor_commander_round_1_second_iter} is now used as the truth in Round 1, which was visualized using Table \ref{fig:traitor_commander_round_2} to get this new Table that represents the knowledge after completion of an iteration of Round 1 when $\gen{2}$ completes actings as the commander.}
\end{center}




\subsection*{\texttt{\large Concluding the Example}}
\justifying
After all lieutenants have completed acting as the commander in Round 1 and obtaining the majority from their respecive Round 0, the final knowledge table for Round 1 is visualized in Table \ref{tab:traitor_commander_final_table}

Every loyal lieutenant will now take the majority of messages that were obtained from all the Round 0 (recorded in the respective column, in out tabular visualization) and use that as the final decision. In our example, the amount of $Retreat$ and $Attack$ messages are the same for all loyal lieutenants hence they will all use a predetermined decision which is $Retreat$ for the algorithm. As guranteed all the loyal lieutenants ($\gen{2}, \gen{3}, \gen{4}, \gen{5}, \gen{6}$) finally concluded to the same decision which prooves the correctness of the algorithm even when the commander $\gen{0}$ itself was a traitor.

\begin{center}
\vspace{1em} % Add some vertical space
\renewcommand{\arraystretch}{1.5}
\setlength{\tabcolsep}{6pt} % Adjusted
\begin{tabular}{c|c|c|c|c|c}
    \hline
    \textbf{$\gen{1}(\traitor)$} & \textbf{$\gen{2}(\loyal)$} & \textbf{$\gen{3}(\loyal)$} & \textbf{$\gen{4}(\loyal)$} & \textbf{$\gen{5}(\loyal)$} & \textbf{$\gen{6}(\loyal)$} \\
    \hline
    \cellcolor{yellow!30}\textbf{\textcolor{red}{\cmdR}} & \cellcolor{red!75}\textbf{\textcolor{white}{R}} & \cellcolor{red!75}\textbf{\textcolor{white}{R}} & \cellcolor{red!75}\textbf{\textcolor{white}{R}} & \cellcolor{red!75}\textbf{\textcolor{white}{R}} & \cellcolor{red!75}\textbf{\textcolor{white}{R}} \\
    \hline
    \cellcolor{blue!75}\textbf{\textcolor{white}{A}} & \cellcolor{yellow!30}\textbf{\textcolor{blue}{\cmdA}} & \cellcolor{blue!75}\textbf{\textcolor{white}{A}} & \cellcolor{blue!75}\textbf{\textcolor{white}{A}} & \cellcolor{blue!75}\textbf{\textcolor{white}{A}} & \cellcolor{blue!75}\textbf{\textcolor{white}{A}} \\
    \hline
    \cellcolor{red!75}\textbf{\textcolor{white}{R}} & \cellcolor{red!75}\textbf{\textcolor{white}{R}} & \cellcolor{yellow!30}\textbf{\textcolor{red}{\cmdR}} & \cellcolor{red!75}\textbf{\textcolor{white}{R}} & \cellcolor{red!75}\textbf{\textcolor{white}{R}} & \cellcolor{red!75}\textbf{\textcolor{white}{R}} \\
    \hline
    \cellcolor{blue!75}\textbf{\textcolor{white}{A}} & \cellcolor{blue!75}\textbf{\textcolor{white}{A}} & \cellcolor{blue!75}\textbf{\textcolor{white}{A}} & \cellcolor{yellow!30}\textbf{\textcolor{blue}{\cmdA}} & \cellcolor{blue!75}\textbf{\textcolor{white}{A}} & \cellcolor{blue!75}\textbf{\textcolor{white}{A}} \\
    \hline
    \cellcolor{red!75}\textbf{\textcolor{white}{R}} & \cellcolor{red!75}\textbf{\textcolor{white}{R}} & \cellcolor{red!75}\textbf{\textcolor{white}{R}} & \cellcolor{red!75}\textbf{\textcolor{white}{R}} & \cellcolor{yellow!30}\textbf{\textcolor{red}{\cmdR}} & \cellcolor{red!75}\textbf{\textcolor{white}{R}} \\
    \hline
    \cellcolor{blue!75}\textbf{\textcolor{white}{A}} & \cellcolor{blue!75}\textbf{\textcolor{white}{A}} & \cellcolor{blue!75}\textbf{\textcolor{white}{A}} & \cellcolor{blue!75}\textbf{\textcolor{white}{A}} & \cellcolor{blue!75}\textbf{\textcolor{white}{A}} & \cellcolor{yellow!30}\textbf{\textcolor{blue}{\cmdA}} \\
    \hline
    \hline
    \cellcolor{red!45}\textbf{\textcolor{white}{R}} & \cellcolor{red!75}\textbf{\textcolor{white}{R}} & \cellcolor{red!75}\textbf{\textcolor{white}{R}} & \cellcolor{red!75}\textbf{\textcolor{white}{R}} & \cellcolor{red!75}\textbf{\textcolor{white}{R}} & \cellcolor{red!75}\textbf{\textcolor{white}{R}} \\
\end{tabular}
\label{tab:traitor_commander_final_table}
\captionof{table}{Final Table}
\end{center}

\par\vspace{0.3em}


\justifying
To summarize, The traitor lieutenant $\gen{1}$ (who received $Retreat$ from $\gen{0}$) continues the deception by:
\begin{itemize}
    \item Telling some lieutenants that it received $Attack$ from the commander
    \item Telling other lieutenants that it received $Retreat$ from the commander
\end{itemize}

Meanwhile, all loyal lieutenants honestly relay the commands they received. After this round, each lieutenant has:
\begin{itemize}
    \item Their original command from the commander, $\gen{0}$ from Round 0.
    \item Reports from all other lieutenants about what they claim to have received
\end{itemize}

The final table shows what each lieutenant knows after all Round 2 and Round 1 messages have been exchanged.
This demonstrates the power of the OM(2) algorithm: despite having a traitor commander and a traitor lieutenant, all loyal lieutenants reach the same conclusion. The Byzantine Generals' Problem is solved for this instance!

\section*{\texttt{\Large A Second Example: Loyal Commander with Collaborating Traitors}}

\justifying
To completely solidify our understanding of the algorithm let's examine a different but also  crucial scenario where the commander is loyal but traitor lieutenants collaborate to disrupt consensus.

\begin{itemize}
    \item \textbf{Total generals}: $N = 7$
    \item \textbf{Traitors}: $t = 2$
    \item \textbf{Loyal generals}: $N - t = 5$
\end{itemize}

In this scenario:
\begin{itemize}
    \item Commander $\gen{0}$ is loyal
    \item Lieutenants $\gen{1}$ and $\gen{4}$ are traitors
    \item Lieutenants $\gen{2}$, $\gen{3}$, $\gen{5}$, and $\gen{6}$ are loyal
\end{itemize}

We'll execute the OM(2) algorithm for this scenario and observe how the loyal lieutenants achieve consensus despite the traitors' attempts to disrupt it.

\subsection*{\texttt{\large Round 0}}
\justifying
In Round 0, the loyal commander $\gen{0}$ sends the same command to all lieutenants. For this example, the commander issues an \textit{Attack} order. Figure~\ref{fig:loyal_commander_round_2} illustrates this initial, honest communication.

% TikZ styles for Byzantine Generals figures
\tikzstyle{general} = [circle, draw, minimum size=0.9cm, font=\small]
\tikzstyle{messageA} = [-{Stealth[length=2.5mm, width=2mm]}, blue, thick] % Blue for Attack
\tikzstyle{messageR} = [-{Stealth[length=2.5mm, width=2mm]}, red, thick]  % Red for Retreat
\tikzstyle{nodeLabel} = [font=\scriptsize, right=1pt]

% Round 2 Figure
\begin{center}
\begin{tikzpicture}[scale=1, transform shape]
    % Commander (now loyal)
    \node[general, fill=green!20, label={[nodeLabel]right:{Commander}}] (g0) at (0,4) {$\gen{0}(\loyal)$};

    % Lieutenants (two traitors: G1 and G4)
    \node[general, fill=red!20] (g1) at (-5,0) {$\gen{1}(\traitor)$};
    \node[general, fill=green!20] (g2) at (-3,0) {$\gen{2}(\loyal)$};
    \node[general, fill=green!20] (g3) at (-1,0) {$\gen{3}(\loyal)$};
    \node[general, fill=red!20] (g4) at (1,0) {$\gen{4}(\traitor)$};
    \node[general, fill=green!20] (g5) at (3,0) {$\gen{5}(\loyal)$};
    \node[general, fill=green!20] (g6) at (5,0) {$\gen{6}(\loyal)$};

    % Messages from G0 - all the same since commander is loyal
    \draw[messageA] (g0) -- (g1) node[midway, above, sloped, black]{\cmdA}; % G0 to G1
    \draw[messageA] (g0) -- (g2) node[midway, above, sloped, black]{\cmdA}; % G0 to G2
    \draw[messageA] (g0) -- (g3) node[midway, above, sloped, black]{\cmdA}; % G0 to G3
    \draw[messageA] (g0) -- (g4) node[midway, above, sloped, black]{\cmdA}; % G0 to G4
    \draw[messageA] (g0) -- (g5) node[midway, above, sloped, black]{\cmdA}; % G0 to G5
    \draw[messageA] (g0) -- (g6) node[midway, above, sloped, black]{\cmdA}; % G0 to G6
\end{tikzpicture}
\captionof{figure}{Round 0, Loyal Commander $\gen{0}$ sends a consistent $Attack$ message to all lieutenants.}
\label{fig:loyal_commander_round_2}
\end{center}

\justifying
Because the commander is loyal, every lieutenant, including the traitors $\gen{1}$ and $\gen{4}$, receives the same \textit{Attack} command. Table~\ref{tab:loyal_commander_round_2_values} shows the initial state of knowledge for each lieutenant after this round. The diagonal entries represent the command each lieutenant received directly from $\gen{0}$.

\begin{center}
\vspace{1em} % Add some vertical space
\renewcommand{\arraystretch}{1.5}
\setlength{\tabcolsep}{6pt} % Adjusted
\begin{tabular}{c|c|c|c|c|c}
    \hline
    \textbf{$\gen{1}(\traitor)$} & \textbf{$\gen{2}(\loyal)$} & \textbf{$\gen{3}(\loyal)$} & \textbf{$\gen{4}(\traitor)$} & \textbf{$\gen{5}(\loyal)$} & \textbf{$\gen{6}(\loyal)$} \\
    \hline
    \cellcolor{yellow!30}\textbf{\textcolor{blue}{\cmdA}} & & & & & \\
    \hline
     & \cellcolor{yellow!30}\textbf{\textcolor{blue}{\cmdA}} & & & & \\
    \hline
     & & \cellcolor{yellow!30}\textbf{\textcolor{blue}{\cmdA}} & & & \\
    \hline
     & & & \cellcolor{yellow!30}\textbf{\textcolor{blue}{\cmdA}} & & \\
    \hline
     & & & & \cellcolor{yellow!30}\textbf{\textcolor{blue}{\cmdA}} & \\
    \hline
     & & & & & \cellcolor{yellow!30}\textbf{\textcolor{blue}{\cmdA}} \\
    \hline
\end{tabular}
\captionof{table}{Initial values received by each lieutenant from the loyal commander. Off-diagonal cells are empty, to be filled by messages exchanged in the next rounds.}
\label{tab:loyal_commander_round_2_values}
\end{center}

\subsection*{\texttt{\large Round 1}}
\justifying
In Round 1, each lieutenant acts as the commander and relays the message they received from $\gen{0}$ to all other lieutenants. This is where the traitors make their move.

\justifying
Let's consider the iteration where traitor $\gen{1}$ is the acting commander. Although $\gen{1}$ received an $Attack$ command, it will lie and tell every other lieutenant that it received a $Retreat$ command. Figure~\ref{fig:loyal_commander_traitor_round_1} illustrates this deception.

\begin{center}
\begin{tikzpicture}[scale=1, transform shape]
    % Commander (loyal)
    \node[general, fill=green!20, label={[nodeLabel]right:{Commander}}] (g0) at (0,4) {$\gen{0}(\loyal)$};

    % Lieutenants (Level 1)
    \node[general, fill=red!20] (g1) at (-5,0) {$\gen{1}(\traitor)$};
    \node[general, fill=green!5] (g2) at (-3,0) {$\gen{2}(\loyal)$};
    \node[general, fill=green!5] (g3) at (-1,0) {$\gen{3}(\loyal)$};
    \node[general, fill=red!5] (g4) at (1,0) {$\gen{4}(\traitor)$};
    \node[general, fill=green!5] (g5) at (3,0) {$\gen{5}(\loyal)$};
    \node[general, fill=green!5] (g6) at (5,0) {$\gen{6}(\loyal)$};

    % Messages from Commander (Level 1 edges)
    \draw[messageA] (g0) -- (g1) node[midway, above, sloped, black]{\cmdA};
    \draw[messageA] (g0) -- (g2) node[midway, above, sloped, black]{\cmdA};
    \draw[messageA] (g0) -- (g3) node[midway, above, sloped, black]{\cmdA};
    \draw[messageA] (g0) -- (g4) node[midway, above, sloped, black]{\cmdA};
    \draw[messageA] (g0) -- (g5) node[midway, above, sloped, black]{\cmdA};
    \draw[messageA] (g0) -- (g6) node[midway, above, sloped, black]{\cmdA};

    % Second Level: Lieutenants communicating with each other
    \begin{scope}[yshift=-3cm] % Shift downward for the second layer
        % New nodes (spaced horizontally like the first layer)
        \node[general, fill=green!20] (g2a) at (-3,-3) {$\gen{2}(\loyal)$};
        \node[general, fill=green!20] (g3a) at (-1,-3) {$\gen{3}(\loyal)$};
        \node[general, fill=red!20] (g4a) at (1,-3) {$\gen{4}(\traitor)$};
        \node[general, fill=green!20] (g5a) at (3,-3) {$\gen{5}(\loyal)$};
        \node[general, fill=green!20] (g6a) at (5,-3) {$\gen{6}(\loyal)$};

        % Draw edges - traitor G1 sends contradictory Retreat messages to all lieutenants
            \draw[messageR] (g1) -- (g2a) node[midway, above, sloped, black]{\cmdR};
            \draw[messageR] (g1) -- (g3a) node[midway, above, sloped, black]{\cmdR};
            \draw[messageR] (g1) -- (g4a) node[midway, above, sloped, black]{\cmdR};
            \draw[messageR] (g1) -- (g5a) node[midway, above, sloped, black]{\cmdR};
            \draw[messageR] (g1) -- (g6a) node[midway, above, sloped, black]{\cmdR};
    \end{scope}
\end{tikzpicture}
\captionof{figure}{Traitor lieutenant $\gen{1}$, acting as commander, falsely reports receiving a \textit{Retreat} command from $\gen{0}$.}
\label{fig:loyal_commander_traitor_round_1}
\end{center}

\subsubsection*{\texttt{\large Round 2}}
\justifying
Now, the other lieutenants ($\gen{2}$, $\gen{3}$, $\gen{4}$, $\gen{5}$, $\gen{6}$) enter Round 2. They gossip about the command they just received from the acting commander, $\gen{1}$. As shown in Table~\ref{tab:loyal_commander_round_0_iter_1}, loyal lieutenants honestly relay the \textit{Retreat} message they received from $\gen{1}$. The other traitor, $\gen{4}$, also relays \textit{Retreat}, collaborating with the other traitor. Each lieutenant then takes the majority of the messages in their column. Since $\gen{1}$ sent a consistent lie, every other lieutenant's majority decision for this iteration is \textit{Retreat}.

\sbox{\mainTableBox}{%
  \renewcommand{\arraystretch}{1.3}%
  \setlength{\tabcolsep}{12pt}%
  \begin{tabular}{c|c|c|c|c|c|c}
    \textbf{} & \textbf{$\gen{1}$} & \textbf{$\gen{2}$} & \textbf{$\gen{3}$} & \textbf{$\gen{4}$} & \textbf{$\gen{5}$} & \textbf{$\gen{6}$} \\
    \hline
    \textbf{$\gen{1}$} & \cellcolor{gray!30} & \cellcolor{gray!30} & \cellcolor{gray!30} & \cellcolor{gray!30} & \cellcolor{gray!30} & \cellcolor{gray!30} \\
    \textbf{$\gen{2}$} & \cellcolor{gray!30} & \cellcolor{yellow!30}\textbf{\textcolor{red}{R}} & \cellcolor{red!20}\textcolor{red}{R} & \cellcolor{red!20}\textcolor{red}{R} & \cellcolor{red!20}\textcolor{red}{R} & \cellcolor{red!20}\textcolor{red}{R} \\
    \textbf{$\gen{3}$} & \cellcolor{gray!30} & \cellcolor{red!20}\textcolor{red}{R} & \cellcolor{yellow!30}\textbf{\textcolor{red}{R}} & \cellcolor{red!20}\textcolor{red}{R} & \cellcolor{red!20}\textcolor{red}{R} & \cellcolor{red!20}\textcolor{red}{R} \\
    \textbf{$\gen{4}$} & \cellcolor{gray!30} & \cellcolor{red!20}\textcolor{red}{R} & \cellcolor{red!20}\textcolor{red}{R} & \cellcolor{yellow!30}\textbf{\textcolor{red}{R}} & \cellcolor{red!20}\textcolor{red}{R} & \cellcolor{red!20}\textcolor{red}{R} \\
    \textbf{$\gen{5}$} & \cellcolor{gray!30} & \cellcolor{red!20}\textcolor{red}{R} & \cellcolor{red!20}\textcolor{red}{R} & \cellcolor{red!20}\textcolor{red}{R} & \cellcolor{yellow!30}\textbf{\textcolor{red}{R}} & \cellcolor{red!20}\textcolor{red}{R} \\
    \textbf{$\gen{6}$} & \cellcolor{gray!30} & \cellcolor{red!20}\textcolor{red}{R} & \cellcolor{red!20}\textcolor{red}{R} & \cellcolor{red!20}\textcolor{red}{R} & \cellcolor{red!20}\textcolor{red}{R} & \cellcolor{yellow!30}\textbf{\textcolor{red}{R}} \\
    \hline\hline
    \textbf{Majority} & \cellcolor{gray!30} & \cellcolor{red!75}\textbf{\textcolor{white}{R}}  & \cellcolor{red!75}\textbf{\textcolor{white}{R}}  & \cellcolor{red!75}\textbf{\textcolor{white}{R}}  & \cellcolor{red!75}\textbf{\textcolor{white}{R}} & \cellcolor{red!75}\textbf{\textcolor{white}{R}} \\
  \end{tabular}%
}
\setlength{\mainTableHeight}{\dimexpr\ht\mainTableBox+\dp\mainTableBox\relax}
\setlength{\mainTableWidth}{\wd\mainTableBox}
\setlength{\senderLabelAreaHeight}{\dimexpr 6\mainTableHeight / 7 \relax}
\setlength{\majorityLabelAreaHeight}{\dimexpr \mainTableHeight - \senderLabelAreaHeight \relax}
\settowidth{\labelcolwidth}{\textbf{Majority}}
\begin{center}
\begin{tabular}{@{}c@{\hspace{1em}}c@{}}
  & {\large \textbf{Receiver}} \\[0.3em]
  \adjustbox{valign=c}{\parbox[t][\mainTableHeight][c]{\labelcolwidth}{\centering \parbox[c][\senderLabelAreaHeight][c]{\labelcolwidth}{\centering\rotatebox[origin=c]{90}{\large\textbf{Sender}}}}} &
  \adjustbox{valign=c}{\usebox{\mainTableBox}}
\end{tabular}
\captionof{table}{Round 2 gossip when $\gen{1}$ is commander. All lieutenants report receiving $Retreat$, leading to a unanimous (but false) majority.}
\label{tab:loyal_commander_round_0_iter_1}
\end{center}

The majority of the messages obtained from Round 2 for the iteration, is then used by the lieutenants for that iteration, illustrated in Table \ref{tab:loyal_commander_round_1_iter_1_summary}

% Add the majority from Round 0 to the knowledge table for Round 1 (iteration for G1)
\begin{center}
\renewcommand{\arraystretch}{1.5}
\setlength{\tabcolsep}{6pt}
\begin{tabular}{c|c|c|c|c|c}
    \hline
    \textbf{$\gen{1}(\traitor)$} & \textbf{$\gen{2}(\loyal)$} & \textbf{$\gen{3}(\loyal)$} & \textbf{$\gen{4}(\traitor)$} & \textbf{$\gen{5}(\loyal)$} & \textbf{$\gen{6}(\loyal)$} \\
    \hline
    \cellcolor{yellow!30}\textbf{\textcolor{blue}{\cmdA}} & \cellcolor{red!75}\textbf{\textcolor{white}{R}} & \cellcolor{red!75}\textbf{\textcolor{white}{R}} & \cellcolor{red!75}\textbf{\textcolor{white}{R}} & \cellcolor{red!75}\textbf{\textcolor{white}{R}} & \cellcolor{red!75}\textbf{\textcolor{white}{R}} \\
    \hline
     & \cellcolor{yellow!30}\textbf{\textcolor{blue}{\cmdA}} & & & & \\
    \hline
     & & \cellcolor{yellow!30}\textbf{\textcolor{blue}{\cmdA}} & & & \\
    \hline
     & & & \cellcolor{yellow!30}\textbf{\textcolor{blue}{\cmdA}} & & \\
    \hline
     & & & & \cellcolor{yellow!30}\textbf{\textcolor{blue}{\cmdA}} & \\
    \hline
     & & & & & \cellcolor{yellow!30}\textbf{\textcolor{blue}{\cmdA}} \\
    \hline
\end{tabular}
\captionof{table}{Majority from Round 2 (when $\gen{1}$ is acting commander) is incorporated into the knowledge table for Round 1.}
\label{tab:loyal_commander_round_1_iter_1_summary}
\end{center}

\subsubsection*{\texttt{\large Round 1, Again}}
\justifying
Next, let's see what happens when a loyal lieutenant, $\gen{2}$, is the acting commander. $\gen{2}$ honestly reports the \textit{Attack} command it received from $\gen{0}$ to all other lieutenants, as shown in Figure \ref{fig:loyal_commander_loyal_round_1}.

\begin{center}
\begin{tikzpicture}[scale=1, transform shape]
    % Commander (loyal)
    \node[general, fill=green!20, label={[nodeLabel]right:{Commander}}] (g0) at (0,4) {$\gen{0}(\loyal)$};

    % Lieutenants (Level 1)
    \node[general, fill=red!5] (g1) at (-5,0) {$\gen{1}(\traitor)$};
    \node[general, fill=green!20] (g2) at (-3,0) {$\gen{2}(\loyal)$};
    \node[general, fill=green!5] (g3) at (-1,0) {$\gen{3}(\loyal)$};
    \node[general, fill=red!5] (g4) at (1,0) {$\gen{4}(\traitor)$};
    \node[general, fill=green!5] (g5) at (3,0) {$\gen{5}(\loyal)$};
    \node[general, fill=green!5] (g6) at (5,0) {$\gen{6}(\loyal)$};

    % Messages from Commander (Level 1 edges)
    \draw[messageA] (g0) -- (g1) node[midway, above, sloped, black]{\cmdA};
    \draw[messageA] (g0) -- (g2) node[midway, above, sloped, black]{\cmdA};
    \draw[messageA] (g0) -- (g3) node[midway, above, sloped, black]{\cmdA};
    \draw[messageA] (g0) -- (g4) node[midway, above, sloped, black]{\cmdA};
    \draw[messageA] (g0) -- (g5) node[midway, above, sloped, black]{\cmdA};
    \draw[messageA] (g0) -- (g6) node[midway, above, sloped, black]{\cmdA};

    % Second Level: Lieutenants communicating with each other
    \begin{scope}[yshift=-3cm] % Shift downward for the second layer
        % New nodes (spaced horizontally like the first layer)
        \node[general, fill=red!20] (g1a) at (-5.3,-3) {$\gen{1}(\traitor)$};
        \node[general, fill=green!20] (g3a) at (-1,-3) {$\gen{3}(\loyal)$};
        \node[general, fill=red!20] (g4a) at (1,-3) {$\gen{4}(\traitor)$};
        \node[general, fill=green!20] (g5a) at (3,-3) {$\gen{5}(\loyal)$};
        \node[general, fill=green!20] (g6a) at (5,-3) {$\gen{6}(\loyal)$};

        % Draw edges - loyal G2 sends consistent Attack messages to all lieutenants
            \draw[messageA] (g2) -- (g1a) node[midway, above, sloped, black]{\cmdA};
            \draw[messageA] (g2) -- (g3a) node[midway, above, sloped, black]{\cmdA};
            \draw[messageA] (g2) -- (g4a) node[midway, above, sloped, black]{\cmdA};
            \draw[messageA] (g2) -- (g5a) node[midway, above, sloped, black]{\cmdA};
            \draw[messageA] (g2) -- (g6a) node[midway, above, sloped, black]{\cmdA};
    \end{scope}
\end{tikzpicture}
\captionof{figure}{Loyal lieutenant $\gen{2}$, acting as commander, honestly reports receiving an $Attack$ command.}
\label{fig:loyal_commander_loyal_round_1}
\end{center}

\subsubsection*{\texttt{\large Round 2, Again}}
\justifying
In the corresponding Round 2, the lieutenants of $\gen{2}$ gossip about the message they received. This time, the traitors $\gen{1}$ and $\gen{4}$ lie, claiming they received \textit{Retreat} from $\gen{2}$, while loyal lieutenants report truthfully. Table~\ref{tab:loyal_commander_round_0_iter_2} shows the messages exchanged. Each loyal lieutenant (columns for $\gen{3}, \gen{5}, \gen{6}$) receives two false $Retreat$ messages (from $\gen{1}$ and $\gen{4}$) but two honest $Attack$ messages from two of the remaining loyal lieutenants ($\gen{3}, \gen{5}, \gen{6}$). The majority for all loyal lieutenants is therefore \textit{Attack}. As illustrated in \ref{tab:loyal_commander_round_0_iter_2}



\sbox{\mainTableBox}{%
  \renewcommand{\arraystretch}{1.3}%
  \setlength{\tabcolsep}{12pt}%
  \begin{tabular}{c|c|c|c|c|c|c}
    \textbf{} & \textbf{$\gen{1}$} & \textbf{$\gen{2}$} & \textbf{$\gen{3}$} & \textbf{$\gen{4}$} & \textbf{$\gen{5}$} & \textbf{$\gen{6}$} \\
    \hline
    \textbf{$\gen{1}$} & \cellcolor{yellow!30}\textbf{\textcolor{blue}{A}} & \cellcolor{gray!30} & \cellcolor{red!20}\textcolor{red}{R} & \cellcolor{red!20}\textcolor{red}{R} & \cellcolor{red!20}\textcolor{red}{R} & \cellcolor{red!20}\textcolor{red}{R} \\
    \textbf{$\gen{2}$} & \cellcolor{gray!30} & \cellcolor{gray!30} & \cellcolor{gray!30} & \cellcolor{gray!30} & \cellcolor{gray!30} & \cellcolor{gray!30} \\
    \textbf{$\gen{3}$} & \cellcolor{blue!20}\textcolor{blue}{A} & \cellcolor{gray!30} & \cellcolor{yellow!30}\textbf{\textcolor{blue}{A}} & \cellcolor{blue!20}\textcolor{blue}{A} & \cellcolor{blue!20}\textcolor{blue}{A} & \cellcolor{blue!20}\textcolor{blue}{A} \\
    \textbf{$\gen{4}$} & \cellcolor{red!20}\textbf{\textcolor{red}{R}} & \cellcolor{gray!30} & \cellcolor{red!20}\textcolor{red}{R} & \cellcolor{yellow!30}\textbf{\textcolor{blue}{A}} & \cellcolor{red!20}\textcolor{red}{R} & \cellcolor{red!20}\textcolor{red}{R} \\
    \textbf{$\gen{5}$} & \cellcolor{blue!20}\textcolor{blue}{A} & \cellcolor{gray!30} & \cellcolor{blue!20}\textcolor{blue}{A} & \cellcolor{blue!20}\textcolor{blue}{A} & \cellcolor{yellow!30}\textbf{\textcolor{blue}{A}} & \cellcolor{blue!20}\textcolor{blue}{A} \\
    \textbf{$\gen{6}$} & \cellcolor{blue!20}\textcolor{blue}{A} & \cellcolor{gray!30} & \cellcolor{blue!20}\textcolor{blue}{A} & \cellcolor{blue!20}\textcolor{blue}{A} & \cellcolor{blue!20}\textbf{\textcolor{blue}{A}} & \cellcolor{yellow!30}\textcolor{blue}{A} \\
    \hline\hline
    \textbf{Majority} & \cellcolor{blue!75}\textbf{\textcolor{white}{A}} & \cellcolor{gray!30} & \cellcolor{blue!75}\textbf{\textcolor{white}{A}}& \cellcolor{blue!75}\textbf{\textcolor{white}{A}} & \cellcolor{blue!75}\textbf{\textcolor{white}{A}} & \cellcolor{blue!75}\textbf{\textcolor{white}{A}} \\
  \end{tabular}%
}
\setlength{\mainTableHeight}{\dimexpr\ht\mainTableBox+\dp\mainTableBox\relax}
\setlength{\mainTableWidth}{\wd\mainTableBox}
\setlength{\senderLabelAreaHeight}{\dimexpr 6\mainTableHeight / 7 \relax}
\setlength{\majorityLabelAreaHeight}{\dimexpr \mainTableHeight - \senderLabelAreaHeight \relax}
\settowidth{\labelcolwidth}{\textbf{Majority}}
\begin{center}
\begin{tabular}{@{}c@{\hspace{1em}}c@{}}
  & {\large \textbf{Receiver}} \\[0.3em]
  \adjustbox{valign=c}{\parbox[t][\mainTableHeight][c]{\labelcolwidth}{\centering \parbox[c][\senderLabelAreaHeight][c]{\labelcolwidth}{\centering\rotatebox[origin=c]{90}{\large\textbf{Sender}}}}} &
  \adjustbox{valign=c}{\usebox{\mainTableBox}}
\end{tabular}
\captionof{table}{Round 2 gossip when $\gen{2}$ is commander. Traitors $\gen{1}$ and $\gen{4}$ send false reports, but loyal lieutenants form a majority for \textit{Attack}.}
\label{tab:loyal_commander_round_0_iter_2}
\end{center}

The Round 1 knowledge table, after the completion of Round 0 where \gen{2} acts as the commander, would look like Table \ref{tab:loyal_commander_round_1_iter_2_summary}

% Add the majority from Round 0 to the knowledge table for Round 1 (iteration for G2)
\begin{center}
\renewcommand{\arraystretch}{1.5}
\setlength{\tabcolsep}{6pt}
\begin{tabular}{c|c|c|c|c|c}
    \hline
    \textbf{$\gen{1}(\traitor)$} & \textbf{$\gen{2}(\loyal)$} & \textbf{$\gen{3}(\loyal)$} & \textbf{$\gen{4}(\traitor)$} & \textbf{$\gen{5}(\loyal)$} & \textbf{$\gen{6}(\loyal)$} \\
    \hline
    \cellcolor{yellow!30}\textbf{\textcolor{blue}{\cmdA}} & \cellcolor{red!75}\textbf{\textcolor{white}{R}} & \cellcolor{red!75}\textbf{\textcolor{white}{R}} & \cellcolor{red!75}\textbf{\textcolor{white}{R}} & \cellcolor{red!75}\textbf{\textcolor{white}{R}} & \cellcolor{red!75}\textbf{\textcolor{white}{R}} \\
    \hline
    \cellcolor{blue!75}\textbf{\textcolor{white}{A}} & \cellcolor{yellow!30}\textbf{\textcolor{blue}{\cmdA}} & \cellcolor{blue!75}\textbf{\textcolor{white}{A}} & \cellcolor{blue!75}\textbf{\textcolor{white}{A}} & \cellcolor{blue!75}\textbf{\textcolor{white}{A}} & \cellcolor{blue!75}\textbf{\textcolor{white}{A}} \\
    \hline
     & & \cellcolor{yellow!30}\textbf{\textcolor{blue}{\cmdA}} & & & \\
    \hline
     & & & \cellcolor{yellow!30}\textbf{\textcolor{blue}{\cmdA}} & & \\
    \hline
     & & & & \cellcolor{yellow!30}\textbf{\textcolor{blue}{\cmdA}} & \\
    \hline
     & & & & & \cellcolor{yellow!30}\textbf{\textcolor{blue}{\cmdA}} \\
    \hline
\end{tabular}
\captionof{table}{Majority from Round 2 (when $\gen{2}$ is acting commander) is incorporated into the knowledge table for Round 1.}
\label{tab:loyal_commander_round_1_iter_2_summary}
\end{center}

\subsection*{\texttt{\large Concluding the Example}}
\justifying
This process repeats for every lieutenant acting as commander in Round 1. Traitors $\gen{1}$ and $\gen{4}$ will always report receiving \textit{Retreat}. Loyal lieutenants $\gen{2}, \gen{3}, \gen{5}, \gen{6}$ will always honestly report receiving \textit{Attack}. After all iterations are complete, Table~\ref{tab:loyal_commander_final_knowledge} shows the final state of knowledge. Each row represents the majority value determined from the corresponding Round 1 iteration.

\begin{center}
\vspace{1em} % Add some vertical space
\renewcommand{\arraystretch}{1.5}
\setlength{\tabcolsep}{6pt} % Adjusted
\begin{tabular}{c|c|c|c|c|c}
    \hline
    \textbf{$\gen{1}(\traitor)$} & \textbf{$\gen{2}(\loyal)$} & \textbf{$\gen{3}(\loyal)$} & \textbf{$\gen{4}(\traitor)$} & \textbf{$\gen{5}(\loyal)$} & \textbf{$\gen{6}(\loyal)$} \\
    \hline
    \cellcolor{yellow!30}\textbf{\textcolor{blue}{\cmdA}} & \cellcolor{red!75}\textbf{\textcolor{white}{R}} & \cellcolor{red!75}\textbf{\textcolor{white}{R}} & \cellcolor{red!75}\textbf{\textcolor{white}{R}} & \cellcolor{red!75}\textbf{\textcolor{white}{R}} & \cellcolor{red!75}\textbf{\textcolor{white}{R}} \\
    \hline
    \cellcolor{blue!75}\textbf{\textcolor{white}{A}} & \cellcolor{yellow!30}\textbf{\textcolor{blue}{\cmdA}} & \cellcolor{blue!75}\textbf{\textcolor{white}{A}} & \cellcolor{blue!75}\textbf{\textcolor{white}{A}} & \cellcolor{blue!75}\textbf{\textcolor{white}{A}} & \cellcolor{blue!75}\textbf{\textcolor{white}{A}} \\
    \hline
    \cellcolor{blue!75}\textbf{\textcolor{white}{A}} & \cellcolor{blue!75}\textbf{\textcolor{white}{A}} & \cellcolor{yellow!30}\textbf{\textcolor{blue}{\cmdA}} & \cellcolor{blue!75}\textbf{\textcolor{white}{A}} & \cellcolor{blue!75}\textbf{\textcolor{white}{A}} & \cellcolor{blue!75}\textbf{\textcolor{white}{A}} \\
    \hline
    \cellcolor{red!75}\textbf{\textcolor{white}{R}} & \cellcolor{red!75}\textbf{\textcolor{white}{R}} & \cellcolor{red!75}\textbf{\textcolor{white}{R}} & \cellcolor{yellow!30}\textbf{\textcolor{blue}{\cmdA}} & \cellcolor{red!75}\textbf{\textcolor{white}{R}} & \cellcolor{red!75}\textbf{\textcolor{white}{R}} \\
    \hline
    \cellcolor{blue!75}\textbf{\textcolor{white}{A}} & \cellcolor{blue!75}\textbf{\textcolor{white}{A}} & \cellcolor{blue!75}\textbf{\textcolor{white}{A}} & \cellcolor{blue!75}\textbf{\textcolor{white}{A}} & \cellcolor{yellow!30}\textbf{\textcolor{blue}{\cmdA}} & \cellcolor{blue!75}\textbf{\textcolor{white}{A}} \\
    \hline
    \cellcolor{blue!75}\textbf{\textcolor{white}{A}} & \cellcolor{blue!75}\textbf{\textcolor{white}{A}} & \cellcolor{blue!75}\textbf{\textcolor{white}{A}} & \cellcolor{blue!75}\textbf{\textcolor{white}{A}} & \cellcolor{blue!75}\textbf{\textcolor{white}{A}} & \cellcolor{yellow!30}\textbf{\textcolor{blue}{\cmdA}} \\
    \hline
    \hline
    \cellcolor{blue!45}\textbf{\textcolor{white}{A}} & \cellcolor{blue!75}\textbf{\textcolor{white}{A}} & \cellcolor{blue!75}\textbf{\textcolor{white}{A}} & \cellcolor{blue!45}\textbf{\textcolor{white}{A}} & \cellcolor{blue!75}\textbf{\textcolor{white}{A}} & \cellcolor{blue!75}\textbf{\textcolor{white}{A}} \\
\end{tabular}
\captionof{table}{Final Round 1 knowledge table for all lieutenants after OM(2) completes. Each cell shows the majority value from the corresponding Round 0, and the diagonal shows the direct message from the commander.}
\label{tab:loyal_commander_final_knowledge}
\end{center}

\justifying
Finally, each loyal lieutenant ($\gen{2}$, $\gen{3}$, $\gen{5}$, and $\gen{6}$) takes a majority vote of the values in its column from Table~\ref{tab:loyal_commander_final_knowledge}. It is quite clear that the majority of messages for each loyal lieutenant (\gen{2}, \gen{3}, \gen{5}, \gen{6}) is same as the original command $Attach$ that the loyal commander \gen{0} sent. Although it irrelevant what the traitor lieutenants (\gen{1}\ \& \ \gen{4}) end up doing, the majority of the messages for them is consistent with the loyal lieutenants.

This demonstrates that when the commander is loyal, OM(2) ensures all loyal lieutenants follow the commander's order, satisfying the \textbf{Validity} requirement. The traitors' lies are filtered out by the majority, and \textbf{Agreement} is achieved among the loyal generals.

\section*{\texttt{\Large Conclusion}}

\justifying
The Byzantine Generals' Problem represents one of the fundamental challenges in distributed computing. Through our examination of the Oral Message algorithm (OM), we've seen how it's possible to achieve consensus even in the presence of malicious actors.

Key insights from our exploration:

\begin{itemize}
    \item With $n$ generals and at most $t$ traitors, consensus requires $n \geq 3t+1$ and $m \geq t$ rounds of communication.
    \item The recursive nature of the algorithm allows loyal generals to filter out inconsistent information.
    \item Even with a traitor commander sending contradictory commands, loyal generals can still reach agreement.
    \item The solution works without requiring any form of message authentication or signatures.
\end{itemize}

This algorithm laid the foundation for many fault-tolerant distributed systems we rely on today, from cloud databases to blockchain networks. Modern consensus protocols like Practical Byzantine Fault Tolerance (PBFT) and those used in blockchain systems are direct descendants of Lamport's original solution.

Understanding these principles not only helps in designing robust distributed systems but also provides insight into the fundamental limits of what's achievable in the presence of Byzantine failures.

\end{document}
