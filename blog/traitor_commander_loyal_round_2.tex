% TikZ styles for Byzantine Generals figures
\tikzstyle{general} = [circle, draw, minimum size=0.9cm, font=\small]
\tikzstyle{messageA} = [-{Stealth[length=2.5mm, width=2mm]}, blue, thick] % Blue for Attack
\tikzstyle{messageR} = [-{Stealth[length=2.5mm, width=2mm]}, red, thick]  % Red for Retreat
\tikzstyle{nodeLabel} = [font=\scriptsize, right=1pt]


% SECTION 1: The TikZ diagram (MUST APPEAR FIRST)
% -----------------------------------------------------------------------------

\begin{center}
\begin{tikzpicture}[scale=1, transform shape]
    % Commander
    \node[general, fill=red!20, label={[nodeLabel]right:{Commander}}] (g0) at (0,4) {$\gen{0}(\traitor)$};

    % Lieutenants (Level 1)
    \node[general, fill=red!5] (g1) at (-5,0) {$\gen{1}(\traitor)$};
    \node[general, fill=green!15] (g2) at (-3,0) {$\gen{2}(\loyal)$};
    \node[general, fill=green!5] (g3) at (-1,0) {$\gen{3}(\loyal)$};
    \node[general, fill=green!5] (g4) at (1,0) {$\gen{4}(\loyal)$};
    \node[general, fill=green!5] (g5) at (3,0) {$\gen{5}(\loyal)$};
    \node[general, fill=green!5] (g6) at (5,0) {$\gen{6}(\loyal)$};

    % Messages from Commander (Level 1 edges)
    \draw[messageR] (g0) -- (g1) node[midway, above, sloped, black]{\cmdR};
    \draw[messageA] (g0) -- (g2) node[midway, above, sloped, black]{\cmdA};
    \draw[messageR] (g0) -- (g3) node[midway, above, sloped, black]{\cmdR};
    \draw[messageA] (g0) -- (g4) node[midway, above, sloped, black]{\cmdA};
    \draw[messageR] (g0) -- (g5) node[midway, above, sloped, black]{\cmdR};
    \draw[messageA] (g0) -- (g6) node[midway, above, sloped, black]{\cmdA};

    % Second Level: Lieutenants communicating with each other (add this section)
    \begin{scope}[yshift=-3cm] % Shift downward for the second layer
        % New nodes (spaced horizontally like the first layer)
        \node[general, fill=red!20] (g1a) at (-5.3,-3) {$\gen{1}(\traitor)$};
        \node[general, fill=green!20] (g3a) at (-1,-3) {$\gen{3}(\loyal)$};
        \node[general, fill=green!20] (g4a) at (1,-3) {$\gen{4}(\loyal)$};
        \node[general, fill=green!20] (g5a) at (3,-3) {$\gen{5}(\loyal)$};
        \node[general, fill=green!20] (g6a) at (5,-3) {$\gen{6}(\loyal)$};

        % Draw edges between lieutenants (example: traitor g1 sends conflicting messages)
        \draw[messageA] (g2) -- (g1a) node[midway, above, sloped, black]{\cmdA};

        \draw[messageA] (g2) -- (g3a) node[midway, above, sloped, black]{\cmdA};
        \draw[messageA] (g2) -- (g4a) node[midway, above, sloped, black]{\cmdA};
        \draw[messageA] (g2) -- (g5a) node[midway, above, sloped, black]{\cmdA};
        \draw[messageA] (g2) -- (g6a) node[midway, above, sloped, black]{\cmdA};
    \end{scope}
\end{tikzpicture}
\captionof{figure}{Round 2: Lieutenants exchange information about what they received from the commander. Loyal lieutenant $\gen{2}$ faithfully relays the original message to all other lieutenants.}
\end{center}

% SECTION 2: The Tables (MUST APPEAR SECOND)
% -----------------------------------------------------------------------------

% --- Step 1: Typeset the main data table into a savebox to measure its dimensions ---
\sbox{\mainTableBox}{%
  \renewcommand{\arraystretch}{1.3}%
  \setlength{\tabcolsep}{12pt}%
  \begin{tabular}{c|c|c|c|c|c}
    \textbf{} & \textbf{$\gen{1}$} & \textbf{$\gen{3}$} & \textbf{$\gen{4}$} & \textbf{$\gen{5}$} & \textbf{$\gen{6}$} \\
    \hline

    \textbf{$\gen{1}$} & \cellcolor{yellow!30}\textbf{\textcolor{blue}{A}} & \cellcolor{red!20}\textcolor{red}{R} & \cellcolor{red!20}\textcolor{red}{R} & \cellcolor{red!20}\textcolor{red}{R} & \cellcolor{red!20}\textcolor{red}{R} \\
    \textbf{$\gen{3}$} & \cellcolor{blue!20}\textcolor{blue}{A} & \cellcolor{yellow!30}\textbf{\textcolor{blue}{A}} & \cellcolor{blue!20}\textcolor{blue}{A} & \cellcolor{blue!20}\textcolor{blue}{A} & \cellcolor{blue!20}\textcolor{blue}{A} \\
    \textbf{$\gen{4}$} & \cellcolor{blue!20}\textcolor{blue}{A} & \cellcolor{blue!20}\textbf{\textcolor{blue}{A}} & \cellcolor{yellow!20}\textcolor{blue}{A} & \cellcolor{blue!20}\textcolor{blue}{A} & \cellcolor{blue!20}\textcolor{blue}{A} \\
    \textbf{$\gen{5}$} & \cellcolor{blue!20}\textcolor{blue}{A} & \cellcolor{blue!20}\textcolor{blue}{A} & \cellcolor{blue!20}\textcolor{blue}{A} & \cellcolor{yellow!30}\textbf{\textcolor{blue}{A}} & \cellcolor{blue!20}\textcolor{blue}{A} \\
    \textbf{$\gen{6}$} & \cellcolor{blue!20}\textcolor{blue}{A} & \cellcolor{blue!20}\textcolor{blue}{A} & \cellcolor{blue!20}\textcolor{blue}{A} & \cellcolor{blue!20}\textbf{\textcolor{blue}{A}} & \cellcolor{yellow!20}\textcolor{blue}{A} \\
    \hline\hline
    \textbf{Majority} & \cellcolor{blue!75}\textbf{\textcolor{white}{A}} & \cellcolor{blue!75}\textbf{\textcolor{white}{A}}& \cellcolor{blue!75}\textbf{\textcolor{white}{A}} & \cellcolor{blue!75}\textbf{\textcolor{white}{A}} & \cellcolor{blue!75}\textbf{\textcolor{white}{A}} \\
  \end{tabular}%
}
% --- Step 2: Get the measured height and width of the saved table ---
\setlength{\mainTableHeight}{\dimexpr\ht\mainTableBox+\dp\mainTableBox\relax}
\setlength{\mainTableWidth}{\wd\mainTableBox}


% --- Step 3: Calculate proportional heights for Sender and Majority label areas ---
% Sender area corresponds to 6 out of 7 conceptual rows.
% Majority area corresponds to 1 out of 7 conceptual rows.
\setlength{\senderLabelAreaHeight}{\dimexpr 6\mainTableHeight / 7 \relax}
\setlength{\majorityLabelAreaHeight}{\dimexpr \mainTableHeight - \senderLabelAreaHeight \relax} % Ensures sum is exactly \mainTableHeight

% --- Step 4: Assemble the labels and table using adjustbox for vertical centering ---

\newlength{\labelcolwidth} % Define a new length
\settowidth{\labelcolwidth}{\textbf{Majority}} % Set it to the width of "Majority"

\begin{center}
\begin{tabular}{@{}c@{\hspace{1em}}c@{}}
  % Top row: Empty space above Sender label + Receiver header
  & {\large \textbf{Receiver}} \\[0.3em]
  % Bottom row: Sender label + Main data table
  \adjustbox{valign=c}{% Vertically center the entire label column box
    \parbox[t][\mainTableHeight][c]{\labelcolwidth}{ % Use \labelcolwidth for the outer parbox width
      \centering % Horizontally center the inner parboxes (Sender & Majority areas)
      \parbox[c][\senderLabelAreaHeight][c]{\labelcolwidth}{ % Sender area, use \labelcolwidth
        \centering\rotatebox[origin=c]{90}{\large\textbf{Sender}}
      }
    }
  }
  &
  \adjustbox{valign=c}{\usebox{\mainTableBox}} % Use the saved (and measured) table, vertically centered
\end{tabular}
\end{center}


\vspace{0.7em}

{\small
\textbf{Legend:} Rows (left) are senders, columns (top) are receivers. Each cell shows the message the sender tells the receiver that it received from the commander. Yellow diagonal: sender's own message from the commander. Blue = Attack (A), Red = Retreat (R).
}

\begin{center}
\vspace{1em} % Add some vertical space
\renewcommand{\arraystretch}{1.5}
\setlength{\tabcolsep}{6pt} % Adjusted
\begin{tabular}{c|c|c|c|c|c}
    \hline
    \textbf{$\gen{1}(\traitor)$} & \textbf{$\gen{2}(\loyal)$} & \textbf{$\gen{3}(\loyal)$} & \textbf{$\gen{4}(\loyal)$} & \textbf{$\gen{5}(\loyal)$} & \textbf{$\gen{6}(\loyal)$} \\
    \hline
    \cellcolor{yellow!30}\textbf{\textcolor{red}{\cmdR}} & \cellcolor{red!75}\textbf{\textcolor{white}{R}} & \cellcolor{red!75}\textbf{\textcolor{white}{R}} & \cellcolor{red!75}\textbf{\textcolor{white}{R}} & \cellcolor{red!75}\textbf{\textcolor{white}{R}} & \cellcolor{red!75}\textbf{\textcolor{white}{R}} \\
    \hline
    \cellcolor{blue!75}\textbf{\textcolor{white}{A}} & \cellcolor{yellow!30}\textbf{\textcolor{blue}{\cmdA}} & \cellcolor{blue!75}\textbf{\textcolor{white}{A}} & \cellcolor{blue!75}\textbf{\textcolor{white}{A}} & \cellcolor{blue!75}\textbf{\textcolor{white}{A}} & \cellcolor{blue!75}\textbf{\textcolor{white}{A}} \\
    \hline
    & & \cellcolor{yellow!30}\textbf{\textcolor{red}{\cmdR}} & & & \\
    \hline
    & & & \cellcolor{yellow!30}\textbf{\textcolor{blue}{\cmdA}} & & \\
    \hline
    & & & & \cellcolor{yellow!30}\textbf{\textcolor{red}{\cmdR}} & \\
    \hline
    & & & & & \cellcolor{yellow!30}\textbf{\textcolor{blue}{\cmdA}} \\
    \hline
\end{tabular}
\end{center}

\par\vspace{0.3em}
{\scriptsize
After Round 2, each loyal lieutenant can determine the consensus by taking a majority vote of all reported values. Despite the conflicting information from the traitors, all loyal lieutenants reach the same conclusion: Retreat (R).
}
