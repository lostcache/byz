% TikZ styles for Byzantine Generals figures
\tikzstyle{general} = [circle, draw, minimum size=0.9cm, font=\small]
\tikzstyle{messageA} = [-{Stealth[length=2.5mm, width=2mm]}, blue, thick] % Blue for Attack
\tikzstyle{messageR} = [-{Stealth[length=2.5mm, width=2mm]}, red, thick]  % Red for Retreat
\tikzstyle{nodeLabel} = [font=\scriptsize, right=1pt]

% Round 1 Figure
$\gen{0}$ sends its chosen messages to the lieutenants.
\begin{figure}[htb]
\centering
\begin{tikzpicture}[scale=1, transform shape]
    % Commander
    \node[general, fill=red!20, label={[nodeLabel]right:{Commander}}] (g0) at (0,4) {$\gen{0}(\traitor)$};

    % Lieutenants
    \node[general, fill=red!20] (g1) at (-5,0) {$\gen{1}(\traitor)$};
    \node[general, fill=green!20] (g2) at (-3,0) {$\gen{2}(\loyal)$};
    \node[general, fill=green!20] (g3) at (-1,0) {$\gen{3}(\loyal)$};
    \node[general, fill=green!20] (g4) at (1,0) {$\gen{4}(\loyal)$};
    \node[general, fill=green!20] (g5) at (3,0) {$\gen{5}(\loyal)$};
    \node[general, fill=green!20] (g6) at (5,0) {$\gen{6}(\loyal)$};

    % Messages from G0
    \draw[messageR] (g0) -- (g1) node[midway, above, sloped, black]{\cmdR}; % G0 to G1 (Traitor)
    \draw[messageA] (g0) -- (g2) node[midway, above, sloped, black]{\cmdA}; % G0 to G2 (Loyal)
    \draw[messageR] (g0) -- (g3) node[midway, above, sloped, black]{\cmdR}; % G0 to G3 (Loyal)
    \draw[messageA] (g0) -- (g4) node[midway, above, sloped, black]{\cmdA}; % G0 to G4 (Loyal)
    \draw[messageR] (g0) -- (g5) node[midway, above, sloped, black]{\cmdR}; % G0 to G5 (Loyal)
    \draw[messageA] (g0) -- (g6) node[midway, above, sloped, black]{\cmdA}; % G0 to G6 (Loyal)
\end{tikzpicture}
\caption{Round 1: Traitor Commander $\gen{0}$ sends inconsistent messages to different lieutenants.}

\begin{center}
\vspace{1em} % Add some vertical space
\renewcommand{\arraystretch}{1.5}
\setlength{\tabcolsep}{6pt} % Adjusted
\begin{tabular}{c|c|c|c|c|c}
    \hline
    \textbf{$\gen{1}(\traitor)$} & \textbf{$\gen{2}(\loyal)$} & \textbf{$\gen{3}(\loyal)$} & \textbf{$\gen{4}(\loyal)$} & \textbf{$\gen{5}(\loyal)$} & \textbf{$\gen{6}(\loyal)$} \\
    \hline
    \cellcolor{yellow!30}\textbf{\textcolor{red}{\cmdR}} & & & & & \\
    \hline
     & \cellcolor{yellow!30}\textbf{\textcolor{blue}{\cmdA}} & & & & \\
    \hline
     & & \cellcolor{yellow!30}\textbf{\textcolor{red}{\cmdR}} & & & \\
    \hline
     & & & \cellcolor{yellow!30}\textbf{\textcolor{blue}{\cmdA}} & & \\
    \hline
     & & & & \cellcolor{yellow!30}\textbf{\textcolor{red}{\cmdR}} & \\
    \hline
     & & & & & \cellcolor{yellow!30}\textbf{\textcolor{blue}{\cmdA}} \\
    \hline
\end{tabular}
\par\vspace{0.3em}
{\scriptsize
The grid above shows the initial values $v_i$ that each lieutenant $\gen{i}$ (represented by row and column headers) receives from Commander $\gen{0}$ and stores. This value is placed on the diagonal cell $(\gen{i}, \gen{i})$. Off-diagonal cells are empty, to be filled by messages exchanged between lieutenants in later algorithm phases (e.g., OM(m-1) rounds).
}
\end{center}

\end{figure}

